\documentclass[sigconf]{acmart}

\usepackage[english]{babel}
\usepackage{blindtext}
\usepackage{subcaption}
\usepackage[]{graphicx}

% Copyright
\renewcommand\footnotetextcopyrightpermission[1]{} % removes footnote with conference info
\setcopyright{none}
%\setcopyright{acmcopyright}
%\setcopyright{acmlicensed}
%\setcopyright{rightsretained}
%\setcopyright{usgov}
%\setcopyright{usgovmixed}
%\setcopyright{cagov}
%\setcopyright{cagovmixed}

\settopmatter{printacmref=false, printccs=false, printfolios=true}

% DOI
\acmDOI{}

% ISBN
\acmISBN{}

%Conference
%\acmConference[Submitted for review to SIGCOMM]{}
%\acmYear{2018}
%\copyrightyear{}

%% {} with no args suppresses printing of the price
\acmPrice{}


\begin{document}
\title{Measurement study of DNS over IPv6: Problem and Solution}

%\titlenote{Produces the permission block, and copyright information}
%\subtitle{Extended Abstract}

\author{Linjian Song, Shengling Wang}
% \author{Firstname Lastname}
% \authornote{Note}
% \orcid{1234-5678-9012}
% \affiliation{%
%   \institution{Affiliation}
%   \streetaddress{Address}
%   \city{City} 
%   \state{State} 
%   \postcode{Zipcode}
% }
% \email{email@domain.com}

% The default list of authors is too long for headers}
\renewcommand{\shortauthors}{X.et al.}

\begin{abstract}
   Internet works on reliable naming and IP package transmission. 
   As complexity added into Internet infrastructure due to security 
   and stability consideration, some mismatch and unexpected fragility 
   are exposed. One case is in the field of DNS (DNSSEC) and IPv6. 
   There are some public evidence and concerns on IPv6 fragmentation 
   issues due to larger DNS payloads over IPv6. Different from other 
   measurement work, in this paper we proposed "glueless" measurement 
   mechanism to analyze the end-to-end users experience worldwide. 
   It is based on a live, real and global Ads system and shows that 
   the IPv6 large packet drop rate is up to 38\%. Our analysis shows 
   the combination of DNSSEC, UDP and IPv6 is really not going to work 
   very well. In the meanwhile, we advance a solution called ATR 
   (Additional Truncated Response) as a fix on DNS server which 
   requires no modification on users side (DNS resolver) and support 
   incremental deployment. We also conducted a measurement which shows 
   more than 68\% impacted users can be relieved on DNS latency and 
   failures due to large DNS response.
\end{abstract}

\maketitle

\section{Introduction}
% one page
Large DNS response is identified as a issue for a long time.  There
is an inherent mechanism defined in [RFC1035] to handle large DNS
response (larger than 512 octets) by indicating (set TrunCation bit)
the resolver to fall back to query via TCP.  Due to the fear of cost
of TCP, EDNS(0) [RFC6891] was proposed which encourages server to
response larger response instead of falling back to TCP.  However, as
the increasing use of DNSSEC and IPv6, there are more public evidence
and concerns on user's suffering due to packets dropping caused by
IPv6 fragmentation in DNS due to large DNS response.

It is observed that some IPv6 network devices like firewalls
intentionally choose to drop the IPv6 packets with fragmentation
Headers[I-D.taylor-v6ops-fragdrop].  [RFC7872] reported more than 30\%
drop rates for sending fragmented packets.  Regarding IPv6
fragmentation issue due to larger DNS payloads in response, one
measurement [IPv6-frag-DNS] reported 35\% of endpoints using
IPv6-capable DNS resolver can not receive a fragmented IPv6 response
over UDP.  Moreover, most of the underlying issues with fragments are
unrevealed due to good redundancy and resilience of DNS.  It is hard
for DNS client and server operators to trace and locate the issue
when fragments are blocked or dropped.  The noticeable DNS failures
and latency experienced by end users are just the tip of the iceberg.

Depending on retry model, the resolver's failing to receive
fragmented response may experience long latency or failure due to
timeout and reties.  One typical case is that the resolver finally
got the answer after several retires and it falls back to TCP after
deceasing the payload size in EDNS0.  To avoid that issue, some
authoritative servers may adopt a policy ignoring the UDP payload
size in EDNS0 extension and always truncating the response when the
response size is large than a expected one.  However one study
[Not-speak-TCP] shows that about 17\% of resolvers in the samples can
not ask a query in TCP when they receive truncated response.  It
seems a dilemma to choose hurting either the users who can not
receive fragments or the users without TCP fallback capacity.  There
is also some voice of "moving all DNS over TCP".  But It is generally
desired that DNS can keep the efficiency and high performance by
using DNS UDP in most of time and fallback as soon as possible to TCP
if necessary for some corner case.

To relieve the problem, this memo introduces an improvement on
DNS responding process by replying an Additional Truncated Response
(ATR) just after a normal large response which is to be fragmented.
Generally speaking ATR provides a way to decouple the EDNS0 and TCP
fallback in which they can work independently according to the server
operator's requirement.  One goal of ATR is to relieve the hurt of
users, both stub and recursive resolver, from the position of server,
both authoritative and recursive server.  It does not require any
changes on resolver and has a deploy-and-gain feature to encourage
operators to implement it to benefit their resolvers.


% \begin{figure}[tp]
% \centering
% \includegraphics{figures/mouse}
% \caption{\blindtext}
% \end{figure}

\section{Related Work(TBD)}

% 1 page including tow parts in related work:

% 1) measurement work

Fragmentation Considered Harmful in IPv4 (and IPv6)[9]

IP Fragmentation Considered Fragile 

Many network operators filter all IPv6 fragments.[11]

two researchers utilized a measurement network to measure 
fragment filtering.  They sent packets, fragmented to the 
minimum MTU of 1280, to 502 IPv6 enabled and reachable 
probes. They found that during any given trial period, 
ten percent of the probes did not receive fragmented packets. [12]

Applications That Rely on Fragmentation (draft-bonica-6man-frag-deprecate) [13]

% 2) solution and fix



\section{Measurement study on Large DNS Response Issue}

%2 page with APNIC's measurement and result
We begin by study IPv6 large DNS response issue using a 
novel measurement approach. Basically, we want to use our 
measurements answer the simple question: How IPv6 DNS 
works worldwide with large DNS response? 

\subsection{A Case Study of Large DNS response}

%0.5 pages

To illustrate this situation, here are two DNS queries, 
both made by a recursive resolver to an authoritative 
name server, both using UDP over IPv6.

Query 1:

\$ dig +bufsize=4096 +dnssec 000-4a4-000a-000a0000-b9ec853b-241-1498607999-2a72134a.ap2.
dotnxdomain.net. @8.8.8.8
139.162.21.135
(MSG SIZE rcvd: 1190)

Query 2:

\$ dig +bufsize=4096 +dnssec 000-510-000a-000a-0000-
b9ec853b-241-1498607999-2a72134a.ap2.
dotnxdomain.net. @8.8.8.8
status: SERVFAIL
(MSG SIZE rcvd: 104)

What we see here are two almost identical DNS queries that have
been passed to Google’s Public DNS service to resolve.
In the first case, the DNS response is 1,190 octets long, and in the
second case the response is 1,346 octets long. The DNS server is an
IPv6-only server, and the underlying host of this name server is configured
with a local maximum packet size of 1,280 octets. Therefore,
in the first case the response being sent to the Google resolver is a
single, unfragmented IPv6 UDP packet, and in the second case the
response is broken into two fragmented IPv6 UDP packets. And it is
this single change that triggers the Google Public DNS Server to provide
the intended answer in the first case, but to return a SERVFAIL
failure notice in response to the fragmented IPv6 response. When the
local Maximum Transmission Unit (MTU) on the server is lifted from
1,280 octets to 1,500 octets, the Google resolver returns the server
DNS response in both cases. 

The only difference in these two responses is IPv6 fragmentation, but
there is perhaps more to it than that.

IP fragmentation in both IPv4 and IPv6 “raises the eyebrows” of
firewalls. Firewalls typically use the information provided in the
transport protocol header of the IP packet to decide whether to admit
or deny the packet. For example, you may see firewall rules admitting
packets using TCP ports 80 and 443 as a way of allowing web traffic
through the firewall filter. For this process to work, the inspected
packet needs to contain a TCP header and use the fields in the header
to match against the filter set. Fragmentation in IP duplicates the IP
portion of the packet header, but the inner IP payload, including the
transport protocol header, is not duplicated in every ensuing packet
fragment. Thus trailing fragments pose a conundrum to the firewall.
Either all trailing fragments are admitted, a situation that has its own
set of consequent risks, or all trailing fragments are discarded, a situation
that also poses connection issues[5].

IPv6 adds a further factor to the picture. In IPv4 every IP packet, fragmented
or not, contains IP fragmentation control fields. In IPv6 these
same fragmentation control fields are included in an IPv6 Extension
Header that is attached only to packets that are fragmented This 8-octet 
Extension Header is placed immediately after the IPv6
packet header in all fragmented packets, meaning that a fragmented
IPv6 packet does not contain the Upper Level Protocol Header starting
at octet offset 40 from the start of the IP packet header. But in
the first packet of this set of fragmented packets, the Upper Level
Protocol Header is chained off the fragmentation header, at byte offset
48, assuming that there is only a Fragmentation Extension Header
in the packet. The implications of this fact are quite significant.
Instead of always looking at a fixed point in a packet to determine its
upper-level protocol, the packet-handling device needs to unravel the
Extension Header chain, raising two rather tough questions. First,
how long is the device prepared to spend unraveling this chain? And
second, would the device be prepared to pass on a packet with an
Extension Header that it did not recognize?

In some cases, implementers of IPv6 equipment have found it simpler
to just drop all IPv6 packets that contain Extension Headers. Some
measurements of this behavior are reported in RFC 7872[6]. This
document reports a 38\% packet-drop rate when sending fragmented
IPv6 query packets to DNS Name servers. But the example provided
previously is in fact the opposite case to that reported in RFC 7872,
and the example illustrates a more conventional case. It’s not the
queries in the DNS that can readily grow to sizes that require packet
fragmentation, but the responses. The relevant question concerns
the anticipated probability of packet drop when sending fragmented
UDP IPv6 packets as responses to DNS queries. To rephrase the question
slightly, how do DNS recursive resolvers fare when the IPv6
response from the server is fragmented?

\subsection{How widespread is this problem?}

For a start, it appears from the example cited here that Google’s
Public DNS resolvers experienced some packet-drop problem when
they passed a fragmented IPv6 response (this problem was noted in
mid-2017, and Google has subsequently corrected it). But was this
problem limited to just one or two DNS resolvers, or do many other
DNS resolvers experience a similar packet-drop issue? How widespread
is this problem? We tested this question using three approaches.
%1.5 pages
\subsection{Repairing Missing "Glue" with Large DNS packets}

The measurement technique we are using is based on scripting inside online ads[TBD]. This allows us to instrument a server and get the endpoints who are executing the measurement script to interact with the server. However, we cannot see what the endpoint is doing. For example, we can see from the server when we deliver a DNS response to a client, but we have no clear way to confirm that the client received the response. Normally the mechanisms are indirect, such as looking at whether or not the client then retrieved a web object that was the target of the DNS name. This measurement approach has some degree of uncertainty, as there are a number of reasons for a missing web object fetch, and the inability to resolve the DNS name is just one of these reasons. Is there a better way to measure how DNS resolvers behave?

The first approach we’ve used here is so-called "Glueless" delegation, a novel measurement mechanism by manipulating dynamic name on DNS name servers. The basic approach is to remove the additional section from the “parent” DNS response that lists the IP address of the authoritative name servers for the delegated “child” domain(Example TBD). A resolver, when provided with this answer must suspend its effort to resolve the original DNS name and instead resolve the name server name. Only when it has completed its task can it resume the original name resolution task. We can manipulate the characteristics of the DNS response from the name server name, and we can confirm if the resolver received the response by observing whether it was then able to resume the original resolution task and query the child name server.

We tested the system using an IPv6-only name server address response that used three response sizes:

Small: 169 octets
Medium: 1,428 octets
Large: 1,886 octets

The local MTU on the server was set to 1,280 octets, so both the medium and large responses were fragmented.

This test was loaded into an online advertising campaign.

Results - I 

68,229,946 experiments 
35,602,243 experiments used IPv6-capable resolvers 


Small:  34,901,983 / 35,602,243 = 98.03\% = 1.97\% Drop 
Medium: 34,715,933 / 35,666,726 = 97.335\% = 2.67\% Drop 
Large:  34,514,927 / 35,787,975 = 96.44\% = 3.56\% Drop 

The first outcome from this data is somewhat surprising. While the overall penetration of IPv6 in the larger Internet is currently some 15\% of users, the DNS resolver result is far higher. Some 52\% of these 68M experiments directed their DNS queries to recursive resolvers that were capable of posing their DNS queries over a IPv6 network substrate.

That's an interesting result:

Some 52\% of tested endpoints use DNS resolvers that are capable of using IPv6.

Interpreting the packet drop probabilities for the three sizes of DNS responses is not so straight forward. There is certainty an increased probability of drop for the larger DNS responses, but this is far lower than the 40\% drop rate reported in RFC 7872.

It seems that we should question the experimental conditions that we used here. Are these responses actually using fragmentation in IPv6?

We observed that a number of recursive resolvers use different query options when resolving the addresses of name servers, as distinct from resolving names. In particular, a number of resolvers, including Google’s public DNS resolvers, strip all EDNS(0) query options from these name server address resolution queries.

When the query has no EDNS(0) options, and in particular when there is no UDP Buffer size option in the query, then the name server responds with what will fit in 512 octets. If the response is larger, and in our case this includes the Medium and Large tests, the name server sets the Truncated Response flag in its response to indicate that the provided response is incomplete. This Truncated Response flag is a signal to the resolver that it should query again, using TCP this time.

In the case of this experiment we saw some 18,571,561 medium-size records resolved using TCP and 19,363,818 large-size records resolved using TCP. This means that the observed rate of failure to resolve the name is not necessarily attributable to an inability to receive fragmented IPv6 UDP packets.

Perhaps if we remove all those instances that use TCP to retrieve the large DNS response, then what do we have left?

UDP-only queries: 

Small:  34,901,983 / 35,602,243 = 98.03\% = 1.97\% Drop 
Medium: 16,238,433 / 17,095,165 = 92.21\% = 5.01\% Drop 
Large:  15,257,853 / 16,424,157 = 93.90\% = 7.10\% Drop 

There is certainly a clearer signal here in this data - some 5\% to 7\% of experiments used DNS resolvers that appeared to be incapable of retrieving a fragmented IPv6 UDP DNS response, as compared to the “base” loss rate as experienced by the control small response of 2\%. Tentatively, we can propose that a minimum of 3\% of clients use DNS resolvers that fail to receive fragmented IPv6 packets.

However, in doing this we have filtered out more than one half of the tests, and perhaps we have filtered out those resolvers that cannot receive IPv6 fragmented packets.

\subsection{Large DNS Packets and Web Fetch}

Our second approach was to use a large response for the ‘final’ response for the requested name.

The way in which this has been done is to pad the response using bogus DNSSEC signature records (RRSIG). These DNSSEC signature records are bogus in the sense that the name itself is not DNSSEC-signed, so the content of the digital signature will never be checked, but as long as the resolver is using EDNS(0) and has turned on the DNSSEC OK bit, which occurs in some 70\% of all DNS queries to authoritative name servers, then the DNSSEC signature records will be added to the response.

We are now looking at the web fetch rate, and looking for a variance between the web fetch rates when the DNS responses involve UDP IPv6 fragmentation. We filtered out all experiments that did not fetch the small DNS web object, all experiments that did not set the DO bit in their query, and all experiments that used TCP for the medium and large experiments. In this case, we are looking for those experiments where a fragmented UDP IPv6 response was passed and testing whether or not the endpoint retrieved the web object.

Results - II 

68,229,946 experiments 
25,096,961 experiments used UDP IPv6-capable resolvers
           and had the DO bit set in the query


Medium: 13,648,884 / 25,096,961 = 54.38\% = 45.62\% Drop 
Large:  13,476,800 / 24,969,527 = 53.97\% = 46.03\% Drop

This is a result that is more consistent with the drop rate reported in RFC 7872, but there are a number of factors at play here, and it is not clear exactly how much of this drop rate can be directly attributed to the issue of packet fragmentation in IPv6 and the network’s handling of IPv6 packets with Extension Headers. Again, there is also the consideration that in only looking at a subset of resolvers, namely those resolvers that use IPv6, use EDNS(0) options and set the DO bit in these queries.

\subsection{Fragmented Small DNS Packets}

Let’s return to the first experiment, as this form of experiment has far less potential sources of noise in the measurement. We are wanting to test whether a fragmented IPv6 packet can be received by recursive DNS resolvers, and our use of a large fragmented response is being frustrated by DNS truncation.

What if we use a customized DNS name server arrangement that gratuitously fragments the small DNS response itself? While the IPv6 specification specifies that network Path MTU sizes should be no smaller than 1,280 octets, it does not specify a minimum size of fragmented IPv6 packets.

The approach we’ve taken in this experiment is to use a user level packet processing system that listens on UDP port 53 and passes all incoming DNS queries to a back-end DNS server. When it receives a response from this back-end server it generates a sequence of IPv6 packets that fragments the DNS payload and uses a raw device socket to pass these packets directly to the device interface.

We are relying on the observation that IPv6 packet fragmentation occurs at the IP level in the protocol stack, so the IPv6 driver at the remote end will reassemble the fragments and pass the UDP payload to the DNS application, and if the payload packets are received by the resolver, there will be no trace that the IPv6 packets were fragmented.

As we are manipulating the response to the query for the address of the name server, we can tell if the recursive resolver has received the fragmented packets if the resolver resumes its original query sequence and queries for the terminal name.

Results - III 

10,851,323 experiments used IPv6 queries for the name server address
 6,786,967 experiments queried for the terminal DNS name

Fragmented Response: 6,786,967 / 10,851,323 = 62.54\% = 37.45\% Drop

This is our second result:

Some 37\% of endpoints used IPv6-capable DNS resolvers that were incapable of receiving a fragmented IPv6 response.

We used three servers for this experiment, on serving Asia Pacific, a second serving the America and the third serving Eurasia and Africa. There are some visible differences in the drop rate:

Asia Pacific: 31\% Drop
Americas: 37\% Drop
Eurasia \& Africa: 47\% Drop

Given that this experiment occurs completely in the DNS, we can track each individual DNS resolver as they query for the name server record then, depending on if they receive the fragmented response, query for the terminal name. There are approximately 2 million recursive resolvers in today’s Internet, but only some 15,000 individual resolvers appear to serve some 90\% of all users. This implies that the behavior of the most intensively used resolvers has a noticeable impact on the overall picture of capabilities if DNS infrastructure for the Internet.

We saw 10,115 individual IPv6 addresses used by IPv6-capable recursive resolvers. Of this set of resolvers, we saw 3,592 resolvers that consistently behaved in a manner that was consistent with being unable to receive a fragmented IPv6 packet, The most intensively used recursive resolvers which exhibit this problem are shown in the following table.

Resolver Hits  AS AS Name, CC
2405:200:1606:672::5    4,178,119      55836 RELIANCEJIO-IN Reliance Jio, IN
2402:8100:c::8 1,352,024   55644 IDEANET1-IN Idea Cellular Limited, IN
2402:8100:c::7 1,238,764   55644 IDEANET1-IN Idea Cellular Limited, IN
2407:0:0:2b::5 938,584  4761  INDOSAT-INP-AP INDOSAT, ID
2407:0:0:2a::3 936,883  4761  INDOSAT-INP-AP INDOSAT, ID
2407:0:0:2a::6 885,322  4761  INDOSAT-INP-AP INDOSAT, ID
2407:0:0:2b::6 882,687  4761  INDOSAT-INP-AP INDOSAT, ID
2407:0:0:2b::2 882,305  4761  INDOSAT-INP-AP INDOSAT, ID
2407:0:0:2a::4 881,604  4761  INDOSAT-INP-AP INDOSAT, ID
2407:0:0:2a::5 880,870  4761  INDOSAT-INP-AP INDOSAT, ID
2407:0:0:2a::2 877,329  4761  INDOSAT-INP-AP INDOSAT, ID
2407:0:0:2b::4 876,723  4761  INDOSAT-INP-AP INDOSAT, ID
2407:0:0:2b::3 876,150  4761  INDOSAT-INP-AP INDOSAT, ID
2402:8100:d::8 616,037  55644 IDEANET1-IN Idea Cellular Limited, IN
2402:8100:d::7 426,648  55644 IDEANET1-IN Idea Cellular Limited, IN
2407:0:0:9::2  417,184  4761  INDOSAT-INP-AP INDOSAT, ID
2407:0:0:8::2  415,375  4761  INDOSAT-INP-AP INDOSAT, ID
2407:0:0:8::4  414,410  4761  INDOSAT-INP-AP INDOSAT, ID
2407:0:0:9::4  414,226  4761  INDOSAT-INP-AP INDOSAT, ID
2407:0:0:9::6  411,993  4761  INDOSAT-INP-AP INDOSAT, ID

This table is slightly misleading in so far as very large recursive resolvers use resolver “farms” and the queries are managed by a collection of query ‘slaves’. We can group these individual resolver IPv6 addresses to their common Origin AS, and look at which networks use resolvers that show this problem with IPv6 Extension Header drops.

The second table (below)now shows the preeminent position of Google’s Public DNS service as the most heavily used recursive resolver, and its Extension Header drop issues, as shown in the example at the start of this article, is consistent with its position at the head of the list of networks that have DNS resolvers with this problem.

AS Hits  \% of Total  AS Name,CC
15169    7,952,272   17.3\% GOOGLE - Google Inc., US
4761  6,521,674   14.2\% INDOSAT-INP-AP INDOSAT, ID
55644 4,313,225   9.4\%  IDEANET1-IN Idea Cellular Limited, IN
22394 4,217,285   9.2\%  CELLCO - Cellco Partnership DBA Verizon Wireless, US
55836 4,179,921   9.1\%  RELIANCEJIO-IN Reliance Jio Infocomm Limited, IN
10507 2,939,364   6.4\%  SPCS - Sprint Personal Communications Systems, US
5650  2,005,583   4.4\%  FRONTIER-FRTR - Frontier Communications, US
2516  1,322,228   2.9\%  KDDI KDDI CORPORATION, JP
6128  1,275,278   2.8\%  CABLE-NET-1 - Cablevision Systems Corp., US
32934 1,128,751   2.5\%  FACEBOOK - Facebook, US
20115 984,165  2.1\%  CHARTER-NET-HKY-NC - Charter Communications, US
9498  779,603  1.7\%  BBIL-AP BHARTI Airtel Ltd., IN
20057 438,137  1.0\%  ATT-MOBILITY-LLC-AS20057 - AT\&T Mobility LLC, US
17813 398,404  0.9\%  MTNL-AP Mahanagar Telephone Nigam Ltd., IN
2527  397,832  0.9\%  SO-NET So-net Entertainment Corporation, JP
45458 276,963  0.6\%  SBN-AWN-AS-02-AP SBN-ISP/AWN-ISP, TH
6167  263,583  0.6\%  Cellco Partnership DBA Verizon Wireless, US
8708  255,958  0.6\%  RCS-RDS 73-75 Dr. Staicovici, RO
38091 255,930  0.6\%  HELLONET-AS-KR CJ-HELLOVISION, KR
18101 168,164  0.4\%  Reliance Communications DAKC MUMBAI, IN



% We used an experiment that tested resolver capabilities in handling
% DNS responses that entailed the use of fragmented UDP IPv6 packets.
% The experiment used a measurement script embedded in an online
% ad to enlist a large number of endpoints to perform resolution of a
% domain name[7]. For this measurement, we altered the DNS resolution
% system to fragment certain DNS responses.

% The approach we took in this experiment was to use a user-level
% packet-processing system that listens on UDP port 53 and passes all
% incoming DNS queries to a back-end DNS server. When it receives
% a response from this back-end server it generates a sequence of IPv6
% packets that fragments the DNS payload and uses a raw device socket
% to pass these packets directly to the device interface.

% We are relying on the observation that IPv6 packet fragmentation
% occurs at the IP level in the protocol stack, so the IPv6 driver at the
% remote end will reassemble the fragments and pass the UDP payload
% to the DNS application, and if the resolver receives the payload packets
% , there will be no trace that the IPv6 packets were fragmented.
% The results of this experiment follow:

% • 10,851,323 experiments used IPv6 queries for the name server
% address.

% • 6,786,967 experiments queried for the terminal DNS name.

% • Fragmented response: 6,786,967/10,851,323 = 62.54\% = 37.45\%
% drop.

% Some 37\% of client endpoints used IPv6-capable DNS resolvers that
% were incapable of receiving a fragmented IPv6 response.

However, one conclusion looks starkly clear to me from these results. We can’t just assume that the DNS as we know it today will just work in an all IPv6 future Internet. We must make some changes in some parts of the protocol design to get around this current widespread problem of IPv6 Extension Header packet loss in the DNS, assuming that we want to have a DNS at all in this all-IPv6 future Internet.

\section{ATR Mechanism}

\subsection{Methodology of ATR}

As we stated in related work, we could move the DNS away from UDP and use TCP instead. That move would certainly make a number of functions a lot easier, including encrypting DNS traffic on the wire as a means of assisting with aspects of personal privacy online as well as accommodating large DNS responses.

However, the downside is that TCP imposes a far greater load overhead on servers, and while it is possible to conceive of an all-TCP DNS environment, it is more challenging to understand the economics of such a move and to understand, in particular, how name publishers and name consumers will share the costs of a more expensive name resolution environment.

If we want to continue to UDP where it’s feasible, and continue to use TCP only as the ‘Plan B’ protocol for name resolution, then can we improve the handling of large response in UDP? Specifically, can we make this hybrid approach — of using UDP when we can, and TCP only when we must — faster and more robust?

An approach to address this challenge is first introduce inIETF draft draft-song-atr-large-resp-00. The approach described in this draft is simple: If a DNS server provides a response that entails sending fragmented UDP packets, then the server should wait for a 10 ms period and also back the original query as a truncated response. If the client receives and reassembles the fragmented UDP response, then the ensuing truncated response will be ignored by the client’s DNS resolver as its outstanding query has already been answered. If the fragmented UDP response is dropped by the network, then the truncated response will be received (as it is far smaller), and reception of this truncated response will trigger the client to switch immediately to re-query using TCP. This behavior is illustrated in Figure 1.

\begin{figure}[tp]
\centering
\includegraphics{figures/mouse}
\caption{ATR behavior}
\end{figure}

\subsection{ATR process or Algorithm}

As show in the Figure 2 the ATR module can be implemented is right after truncation loop if the packet is not going to be fragmented.

\begin{figure}[tp]
\centering
\includegraphics{figures/mouse}
\caption{ATR module in DNS response process}
\end{figure}

The ATR responding process goes as follows:

   o  When an authoritative server receives a query and enters the
      responding process, it first go through the normal truncation loop
      to see whether the size of response surpasses the EDNS0 payload
      size.  If yes, it ends up with responding a truncated packets.  If
      no, it enters the ATR module.

   o  In ATR module, similar like truncation loop, the size of response
      is compared with a value called ATR payload size.  If the response
      of a query is larger than ATR payload size, the server firstly
      sends the normal response and then coin a truncated response with
      the same ID of the query.

   o  The server can reply the coined truncated response in no time.
      But considering the possible impact of network reordering, it is
      suggested a timer to delay the second truncated response, for
      example 10~50 millisecond which can be configured by local
      operation.

   Note that the choice of ATR payload size and timer SHOULD be
   configured locally.  And the operational consideration and guidance
   is discussed in Section 4.2 and Section 4.1 respectively.

   There are three typical cases of ATR-unaware resolver behavior when a
   resolver send query to an ATR server in which the server will
   generate a large response with fragments:

   o  Case 1: a resolver (or sub-resolver) will receive both the large
   response and a very small truncated response in sequence.  It will
   happily accepts the first response and drop the second one because
   the transaction is over.

    o  Case 2: In case a fragment is dropped in the middle, the resolver
   will end up with only receiving the small truncated response.  It
   will retry using TCP in no time.

   o  Case 3: For those (probably 30\%*17\% of them) who can not speak TCP
      and sitting behind a firewall stubbornly dropping fragments.  Just
      say good luck to them!


   In the case authoritative server truncated all response surpass
   certain value , for example setting IPv6-edns-size to 1220 octets,
   ATR will helpful for resolver with TCP capacity, because the resolver
   still has a fair chance to receive the large response.

   \subsection{ATR timer}

   TBD

   \subsection{ATR payload size}
      Regarding the operational choice for ATR payload size, there are some
   good input from APNIC study [scoring-dns-root]on how to react to
   large DNS payload for authoritative server.  The difference in ATR is
   that ATR focuses on the second response after the ordinary response.

   For IPv4 DNS server, it is suggested the study that do not truncate
   and fragment IPv4 UDP response with a payload up to 1472 octets which
   is Ethernet MTU(1500) minus the sum of IPv4 header(20) and UDP
   header(8).  The reason is to avoid gratuitously fragmenting outbound
   packets and TCP fallback at the source.

   In the case of ATR, the first ordinary response is emitted without
   knowing it be to fragmented or not on the path.  If a large value is
   set up to 1472 octets, payload size between 512 octets and the large
   value size will probably get fragmented by aggressive firewalls which
   leads losing the benefit of ATR.  If ATR payload size set exactly 512
   octets, in most of case ATR response and the single unfragmented
   packets are under a race at the risk of RO.

   Given IPv4 fragmentation issue is not so serious compared to IPv6, it
   is suggested in this memo to set ATR payload size 1472 octets which
   means ATR only fit large DNS response larger than 1500 octets in
   IPv4.

   For IPv6 DNS server, similar to IPv4, the APNIC study is suggested
   that do not truncate IPv6 UDP packets with a payload up to 1,452
   octets which is Ethernet MTU(1500) minus the sum of IPv6 header(40)
   and UDP header(8). 1452 octets is chosen to avoid TCP fallback in the
   context that most TCP MSS in the root server is not set probably at
   that time.

   In the case of ATR considering the second truncated response, a
   smaller size: 1232 octets, which is IPv6 MTU for most network
   devices(1280) minus the sum of IPv6 header(40) and UDP header(8),
   should be chosen as ATR payload size to trigger necessary TCP
   fallback.  As a complementary requirement with ATR, the TCP MSS
   should be set 1220 octets to avoid Packet Too Big ICMP message as
   suggested in the APNIC study.

   In short, it is recommended that in IPv4 ATR payload size SHOULD be
   1472 octets, and in IPv6 the value SHOULD be 1232 octets.

   \subsection{Less aggressiveness of ATR}

   There is a concern ATR sends TC=1 response too aggressively
   especially in the beginning of adoption.  ATR can be implemented as
   an optional and configurable feature at the disposal of authoritative
   server operator.  One of the idea to mitigate this aggressiveness,
   ATR may respond TC=1 responses at a low possibility, such as 10%.

   Another way is to reply ATR response selectively.  It is observed
   that RO and IPv6 fragmentation issues are path specific and
   persistent due to the Internet components and middle box.  So it is
   reasonable to keep a ATR "Whitelist" by counting the retries and
   recording the IP destination address of that large response causing
   many retires.  ATR only acts to those queries from the IP address in
   the white list.
 
\section{ATR Evaluation}


   It is worth of mentioning APNIC report[How-ATR-Work] on "How well
   does ATR actually work?" done by Geoff Huston and Joao Damas after 00
   version of ATR draft.  It was reported firstly in IEPG meeting before
   IETF 101 and then posted in APNIC Blog later.

   It is said the test was performed over 55 million endpoints, using an
   on-line ad distribution network to deliver the test script across the
   Internet.  The result is positive that ATR works!  From the end
   users' perspective, in some 9\% of IPv4 cases the use of ATR by the
   server will improve the speed of resolution of a fragmented UDP
   response by signaling to the client an immediate switch to TCP to
   perform a re-query.  The IPv6 behavior would improve the resolution
   times in 15\% of cases.

   It also analyzed the pros and cons of ATR.  On one hand, It is said
   that ATR certainly looks attractive if the objective is to improve
   the speed of DNS resolution when passing large DNS responses.  And
   ATR is incrementally deployable in favor of decision made by each
   server operator.  On another hand, ATR also has some negative
   factors.  One issue is adding another DNS DDoS attack vector due to
   the additional packet sent by ATR, (author's note : very small adding
   actually.)  Another issue is risk of RO by the choice of the delay
   timer which is discussed fully in Section 4.1.

   As a conclusion, it is said that "ATR does not completely fix the
   large response issue.  If a resolver cannot receive fragmented UDP
   responses and cannot use TCP to perform DNS queries, then ATR is not
   going to help.  But where there are issues with IP fragment
   filtering, ATR can make the inevitable shift of the query to TCP a
   lot faster than it is today.  But it does so at a cost of additional
   packets and additional DNS functionality".  "If a faster DNS service
   is your highest priority, then ATR is worth considering", said at the
   end of this report

% 2 page from APNIC's ATR measuring article

\section{Network Re-ordering Consideration}

%Atlas analysis

As introduced in Section 2 ATR timer is a way to avoid the impact of
network reordering(RO).  The value of the timer is critical, because
if the delay is too short, the ATR response may be received earlier
than the fragmented response (the first piece), the resolver will
fall back to TCP bearing the cost which should have been avoided.  If
the delay is too long, the client may timeout and retry which negates
the incremental benefit of ATR.  Generally speaking, the delay of the
timer should be "long enough, but not too long".

To the best knowledge of author, the nature of RO is characterized as
follows hopefully helping ATR users understand RO and how to operate
ATR appropriately in RO context.

o  RO is mainly caused by the parallelism in Internet components and
   links other than network anomaly [Bennett].  It was observed that
   RO is highly related to the traffic load of Internet components.
   So RO will long exists as long as the traffic load continue
   increase and the parallelism is used to enhance network
   throughput.

o  The probability of RO varies largely depending on the different
   tests samples.  Some work shown RO probability below 2% [Paxson]
   [Tinta] and another work was above 90% [Bennett].  But it is
   agreed that RO is site-dependent and path-dependent.  It is
   observed in that when RO happens, it is mostly exhibited
   consistently in a small percentages of the paths.  It is also
   observed that higher rates smaller packets were more prone to RO
   because the sending inter-spacing time was small.

o  It was reported that the inter-arrival time of RO varies from a
   few milliseconds to multiple tens of milliseconds [Tinta].  And
   the larger the packet the larger the inter-arrival time, since
   larger packets will take longer to be transmitted.

Reasonably we can infer that firstly RO should be taken into account
because it long exists due to middle Internet components which can
not be avoided by end-to-end way.  Secondly the mixture of larger and
small packets in ATR case will increase the inter-arrival time of RO
as well as the its probability.  The good news is that the RO is
highly site specific and path specific, and persistent which means

the ATR operator is able to identify a few sites and paths, setup a
tunable timer setting for them, or just put them into a blacklist
without replying ATR response.

Based on the above analysis it i

\bibliographystyle{ACM-Reference-Format}
\bibliography{reference}

\end{document}