\section{Introduction}

   Large DNS response is identified as a issue for a long time.  There
   is an inherent mechanism defined in [RFC1035] to handle large DNS
   response (larger than 512 octets) by indicating (set TrunCation bit)
   the resolver to fall back to query via TCP.  Due to the fear of cost
   of TCP, EDNS(0) [RFC6891] was proposed which encourages server to
   response larger response instead of falling back to TCP.  However, as
   the increasing use of DNSSEC and IPv6, there are more public evidence
   and concerns on user's suffering due to packets dropping caused by
   IPv6 fragmentation in DNS due to large DNS response.

   It is observed that some IPv6 network devices like firewalls
   intentionally choose to drop the IPv6 packets with fragmentation
   Headers[I-D.taylor-v6ops-fragdrop].  [RFC7872] reported more than 30%
   drop rates for sending fragmented packets.  Regarding IPv6
   fragmentation issue due to larger DNS payloads in response, one
   measurement [IPv6-frag-DNS] reported 35% of endpoints using
   IPv6-capable DNS resolver can not receive a fragmented IPv6 response
   over UDP.  Moreover, most of the underlying issues with fragments are
   unrevealed due to good redundancy and resilience of DNS.  It is hard
   for DNS client and server operators to trace and locate the issue
   when fragments are blocked or dropped.  The noticeable DNS failures
   and latency experienced by end users are just the tip of the iceberg.
   
   Depending on retry model, the resolver's failing to receive
   fragmented response may experience long latency or failure due to
   timeout and reties.  One typical case is that the resolver finally
   got the answer after several retires and it falls back to TCP after
   deceasing the payload size in EDNS0.  To avoid that issue, some
   authoritative servers may adopt a policy ignoring the UDP payload
   size in EDNS0 extension and always truncating the response when the
   response size is large than a expected one.  However one study
   [Not-speak-TCP] shows that about 17% of resolvers in the samples can
   not ask a query in TCP when they receive truncated response.  It
   seems a dilemma to choose hurting either the users who can not
   receive fragments or the users without TCP fallback capacity.  There
   is also some voice of "moving all DNS over TCP".  But It is generally
   desired that DNS can keep the efficiency and high performance by
   using DNS UDP in most of time and fallback as soon as possible to TCP
   if necessary for some corner case.

   To relieve the problem, this memo introduces an small improvement on
   DNS responding process by replying an Additional Truncated Response
   (ATR) just after a normal large response which is to be fragmented.
   Generally speaking ATR provides a way to decouple the EDNS0 and TCP
   fallback in which they can work independently according to the server
   operator's requirement.  One goal of ATR is to relieve the hurt of
   users, both stub and recursive resolver, from the position of server,
   both authoritative and recursive server.  It does not require any
   changes on resolver and has a deploy-and-gain feature to encourage
   operators to implement it to benefit their resolvers.

   [REMOVE BEFORE PUBLICATION] Note that ATR is not just a proposed
   idea.  Some advocates of ATR implemented it based on BIND9.  And
   Some verify it based on an large-scale experiment platform of APNIC
   lab Section 3 which is introduced in this memo.

\begin{figure}[tp]
\centering
\includegraphics{figures/mouse}
\caption{\blindtext}
\end{figure}

\section{Related Work}
\blindtext

And we need some citation here\cite{floyd1993random, stoica2001chord}

\Blindtext

\section{System Design}

\subsection{The First Layer}
\Blindtext

\subsection{The Second Layer}
\Blindtext

\section{Evaluation}
\Blindtext

\section{Conclusion}
\blindtext

